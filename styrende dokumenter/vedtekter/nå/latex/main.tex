\documentclass[12pt,a4paper,norsk]{article}
\usepackage[utf8]{inputenc}
\usepackage[T1]{fontenc}
\usepackage{babel}
\usepackage{units}
\usepackage{dirtytalk}
\usepackage{enumerate}
\usepackage{graphicx}

\begin{titlepage}
\begin{center}
    \vspace*{1cm}

    \Huge
    \textbf{Vedtekter for UiO Gaming}
    \vspace{1cm}
    \vspace{3cm}
    \includegraphics[width=0.5\textwidth]{images/logo.jpg}
    \vspace{7cm}

    \LARGE
    Sist revidert:\\
    23.02.2022
\end{center}
\end{titlepage}

\renewcommand{\thesection}{§\arabic{section}}

\begin{document}


\section{Foreningens navn}
Foreningens navn er \say{UiO Gaming}.


\section{Formål}

UiO Gaming sitt formål er å fremme gaming, e-sport, brettspill, og andre spillrelaterte
interesser/kulturer ved UiO. I tillegg til dette vil vi lage et sosialt miljø som tilrettelegger alle
former for aktivitet rundt spill. Foreningen har derfor som mål å arrangere spillkvelder,
turneringer, LAN og andre spillrelaterte aktiviteter åpent for alle som er interesserte.



\section{Organisasjonsform}

UiO Gaming er en frivillig studentorganisasjon ved Universitetet i Oslo og ble stiftet
11.09.2020. UiO Gaming er politisk og religiøst uavhengig.
\\
\\
Foreningen er en frittstående juridisk person med medlemmer og er selveiende. At den er
selveiende innebærer at ingen, verken medlemmer eller andre, har krav på foreningens formue
eller eiendeler, eller er ansvarlig for gjeld eller andre.



\section{Medlemmer}

Et medlemskap gjelder fra 1. januar til 31. desember og koster 50 kroner. Minimum 75\% av 
foreningens medlemmer må være nåværende studenter ved et SiO-registrert lærested eller 
tidligere studenter ved Universitetet i Oslo, og minst 50\% av medlemmene må være semesterregistrert 
ved UiO gjeldende semester. Foreningen har en aldersgrense på fylte 18 år og medlemmer av foreningen må 
fylle et av disse to kravene: være registrert student eller under 35 år.



\section{Rettigheter og plikter knyttet til medlemskapet}

Alle medlemmer har rett til å delta på generalforsamling, har stemmerett og er valgbare til tillitsverv i
foreningen. Medlemmene plikter å forholde seg til vedtak som er fattet av generalforsamling.
\\
\\
Medlemmer som påfører foreningen økonomiske tap i form av bøter eller sanksjoner som følge 
av handlinger som enkelt kan unngås stilles selv til økonomisk ansvar for disse. Styret kan med \nicefrac{2}{3} flertall vedta om medlem(ene) det gjelder skal stilles økonomisk til ansvar eller ikke.



\section{Æresmedlemskap}

Styret kan hvert semester utnevne opp til 2 personer til æresmedlemmer av foreningen, på bakgrunn av at disse har gjort seg særlig fortjent til dette. Æresmedlemmer av foreningen vil få livsvaring medlemskap i foreningen. Æresmedlemmer faller ikke innenfor de normale bestemmelser i §4.



\section{Generalforsamling}

Generalforsamlingen er foreningens øverste organ.
\\
\\
Generalforsamlingen skal holdes minimum en gang i semesteret. Den skal holdes innen utgangen av
desember i høstsemesteret og utgangen av juni på vårsemesteret.
\\
\\
Generalforsamling fatter vedtak ved simpelt flertall. Ingen medlemmer har mer enn en stemme
og stemmegivning kan ikke skje ved fullmakt.
\\
\\
Ekstraordinær generalforsamling kan innkalles hvis minimum \nicefrac{1}{3} av medlemmene krever det
eller styret finner det nødvendig.
\\
\\
Innkalling til generalforsamling skal være medlemmene i hende minimum to uker før
generalforsamlingen. Sakspapirer skal være tilgjengelige for medlemmene minimum en uke før
generalforsamlingen.
\\
\\
Vedtektsendringer som skal stemmes over på generalforsamlingen må være styret i hende
minimum en uke før generalforsamlingen og vedtektsendringer kan bare skje når \nicefrac{2}{3} av
medlemmene stemmer for endringen.



\section{Generalforsamlingens oppgaver}

Generalforsamlingen skal

\begin{itemize}
    \item behandle styrets/leders beretning
    \item behandle regnskap fra forrige semester
    \item behandle budsjett for kommende semester
    \item behandle innkomne forslag
    \item fastsette kontingent
    \item velge valgkomité
    \item velge styremedlemmer
\end{itemize}



\section{Valg}

Det velges nytt styre på ordinær generalforsamling. Hvert styremedlem velges for én periode av gangen.
En valgperiode innefatter tiden mellom to ordinære generalforsamlinger.
\\
\\
Leder og nestleder velges for to perioder, nestleder velges på 
generalforsamling høst og leder generalforsamling vår.
\\
\\
Alle valg ved generalforsamling foretas ved skriftlig votering, dersom det ikke fremsettes forslag
om annet. Hvis ingen oppnår minst halvparten av stemmene, skal en ny avstemning avholdes
mellom de to kandidatene som fikk flest stemmer i første runde. Ved stemmelikhet holdes det
først et nytt valg. Om dette igjen ender i stemmelikhet, velges personen ved loddtrekning.
\\
\\
Dersom et verv ikke blir fylt under generalforsamlingen kan styret gis makt til å supplere 
seg selv ved \nicefrac{2}{3} flertall på styremøtet. Ved ektraordinær generalforsamling kan denne personen kan benkes ved minst \nicefrac{2}{3} flertall.



\section{Valgkomité}

Valgkomiteen velges på generalforsamlingen etter innstilling fra styret, og skal
legge frem innstilling på kandidater til alle øvrige tillitsverv som skal velges på
generalforsamlingen. Medlem av valgkomité som selv blir kandidat til verv,
plikter å tre ut av valgkomiteen med mindre vedkommende skriftlig meddeler
valgkomiteen og forslagsstiller at vedkommende ikke er aktuell for vervet.



\section{Styret}

UiO Gaming består av et styre på fem eller syv personer. I tillegg til faste
medlemmer kan generalforsamlingen velge opptil 5 varaer. Minimum 50\% av styrets
medlemmer må være semesterregistrert ved UiO.
\\
\\
Foreningens styre består av følgende roller:
\begin{itemize}
    \item Leder
    \begin{itemize}
        \item Innkaller og leder styremøter. Har ansvar for kommunikasjonen med eksterne parter,
        samarbeidspartnere, fakultetsstyret og leverandører. Har signaturrett. Det er leders
        overordnede ansvar å sørge for at alle personer i styret jobber etter foreningens formål og deres
        konkrete stillingsbeskrivelser, samt ivaretar foreningens gode rykte.
    \end{itemize}

    \item Nestleder
    \begin{itemize}
        \item Leders stedfortreder. Referent ved styremøter. Ansvar for innkalling og gjennomføring
        av generalforsamling.
    \end{itemize}

    \item Økonomiansvarlig
    \begin{itemize}
        \item Ansvarlig for sunn økonomisk drift av foreningen. Legge frem budsjetter og regnskap ved
        hver generalforsamling. Ansvar for bilagsføring og foreningens konto. Økonomiansvarlig har
        ansvar for oppfølging av alle økonomiske saker og plikter å informere styret fortløpende om
        økonomibruken.
    \end{itemize}
\end{itemize}
\leavevmode

Styret i UiO Gaming er beslutningsdyktige når \nicefrac{1}{2} av styremedlemmene er til stede. Saker i styret vedtas ved simpelt flertall. Ved stemmelikhet har leder dobbeltstemme.



\section{Mistillit}

\begin{enumerate}[a)]
    \item Foreningens medlemmer jf. §4 kan fremme mistillitsforslag mot styremedlemmer og andre tillitsvalgte som er valgt i henhold til §9.
    \item Et slikt forslag vil bli behandlet på en ekstraordinær generalforsamling, hvor krav for gjennomførelse er beskrevet i §7, eller på ordinær generalforsamling dersom denne skal avholdes innen to uker.
    \item Mistillitforslaget vedtas med \nicefrac{2}{3} aktivt flertall.
    \item Dersom et mistillitsforslag blir vedtatt kan generalforsamlingen vedta å holde nyvalg for vervet med periode frem til neste ordinære generalforsamling, eller gi styret makt til å supplere seg selv.
\end{enumerate}



\section{Signaturrett}

Styrets leder, eller nestleder i dens fravær, eller den styrets leder bemyndiger, har signaturrett
på vegne av foreningen ved avtaler med eksterne parter som binder foreningen på noen som
helst måte.



\section{Økonomi}

Foreningens midler skal kun brukes i henhold til foreningens formål.
\\
\\
Det skal alltid foreligge budsjett og regnskap for foreningens midler, med tilhørende bilag.
Budsjett og regnskap gjennomgås og godkjennes på generalforsamling.
\\
\\
Utbetalinger skal alltid godkjennes av leder - eller leders stedfortreder - i tillegg til
økonomiansvarlig.
\\
\\
Ved stemming over økonomiske investeringer trengs det et \nicefrac{2}{3} flertall i styret.



\section{Discord}

Som kommunikasjonskanal for medlemmer skal discordserveren \say{UiO Gaming} benyttes.
Eierskap av serveren faller til en bruker styrt i fellesskap av leder og nestleder. Videre administrasjon av
serveren utføres av styret.



\section{Oppløsning}

Foreningen kan oppløses ved at \nicefrac{2}{3} av de stemmeberettigede medlemmene på
generalforsamlingen stemmer for oppløsning. Sak om oppløsning må være meldt inn til
ordinære saksfrister i forkant av generalforsamlingen. Ved opphør vil foreningens midler og
eiendeler bli gitt til en annen egnet studentforening med samme formål, som
generalforsamlingen bestemmer.
\\
\\
Ingen medlemmer har krav på foreningens midler eller andel av disse.

\section{Etiske Retningslinjer}

\begin{itemize}
    \item LIKEVERD OG INKLUDERING:  
        Alle medlemmer i UiO Gaming skal vise respekt for alle mennesker, 
        uavhengig av alder, kjønn og kjønnsidentitet, religion, seksuell legning, funksjonsevne, sosial 
        status, etnisk tilhørighet eller politisk ståsted. UiO Gaming tar avstand fra alle former for 
        negativ diskriminering, trakassering og mobbing. Vi skal behandle alle med respekt, og avstå 
        fra alle former for kommunikasjon, språkbruk, handling eller behandling som kan oppleves 
        støtende, trakasserende, diskriminerende eller på andre måter ubehagelig.
    
    \item RESPEKT FOR ANDRES GRENSER: 
        UiO Gaming skal være en trygg forening hvor den enkeltes integritet og grenser respekteres. 
        Seksuell trakassering, seksuell overskridende atferd og seksuelle overgrep aksepteres ikke.
     
    \item ÅPENHET OG ÆRLIGHET: 
    Styret i UiO Gaming skal utvise åpenhet i avgjørelser. Referat og vedtak skal som 
    hovedregel være allment tilgjengelige. Alt arbeid skal utføres uten uærlighet og uredelige hensikter.
     
    \item ROLLEFORSTÅELSE: 
    Styret i UiO Gaming skal være bevisst den makt de har i relasjon til andre medlemmer, 
    og ikke misbruke den tillit de har i kraft av sin stilling.
    \end{itemize}

\section{Eksklusjon}

UiO Gaming har rett til å ekskludere medlemmer som bryter med foreningens formål, vedtekter eller de etiske retningslinjene fastsatt i §17. Eksklusjonssaken skal behandles på en ekstraordinær generalforsamling eller et særskilt eksklusjonsmøte innkalt av styret for styret og eventuelle pårørende. Medlemmet det gjelder, skal gis mulighet til å forsvare seg i møtet. Eksklusjonen kan vedtas med \nicefrac{2}{3} aktivt flertall på generalforsamlingen eller eksklusjonsmøtet. 
\\
\\
Ved eksklusjon fratar foreningen medlemmet alle rettigheter og medlemsfordeler, og vedkommende mister også medlemskapets økonomiske forpliktelser og krav til foreningen. Medlemmet som er blitt ekskludert, har krav på tilbakebetaling av medlemskontingenten. Den ekskluderte personen har mulighet til å anke beslutningen om eksklusjonen, med en frist på 2 uker. Styret kan ved en senere anledning med \nicefrac{2}{3} aktivt flertall benåde medlemmet.

\end{document}
